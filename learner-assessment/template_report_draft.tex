\documentclass[]{article}
\usepackage{lmodern}
\usepackage{amssymb,amsmath}
\usepackage{ifxetex,ifluatex}
\usepackage{fixltx2e} % provides \textsubscript
\ifnum 0\ifxetex 1\fi\ifluatex 1\fi=0 % if pdftex
  \usepackage[T1]{fontenc}
  \usepackage[utf8]{inputenc}
\else % if luatex or xelatex
  \ifxetex
    \usepackage{mathspec}
  \else
    \usepackage{fontspec}
  \fi
  \defaultfontfeatures{Ligatures=TeX,Scale=MatchLowercase}
\fi
% use upquote if available, for straight quotes in verbatim environments
\IfFileExists{upquote.sty}{\usepackage{upquote}}{}
% use microtype if available
\IfFileExists{microtype.sty}{%
\usepackage{microtype}
\UseMicrotypeSet[protrusion]{basicmath} % disable protrusion for tt fonts
}{}
\usepackage[margin=1in]{geometry}
\usepackage{hyperref}
\hypersetup{unicode=true,
            pdfborder={0 0 0},
            breaklinks=true}
\urlstyle{same}  % don't use monospace font for urls
\usepackage{longtable,booktabs}
\usepackage{graphicx,grffile}
\makeatletter
\def\maxwidth{\ifdim\Gin@nat@width>\linewidth\linewidth\else\Gin@nat@width\fi}
\def\maxheight{\ifdim\Gin@nat@height>\textheight\textheight\else\Gin@nat@height\fi}
\makeatother
% Scale images if necessary, so that they will not overflow the page
% margins by default, and it is still possible to overwrite the defaults
% using explicit options in \includegraphics[width, height, ...]{}
\setkeys{Gin}{width=\maxwidth,height=\maxheight,keepaspectratio}
\IfFileExists{parskip.sty}{%
\usepackage{parskip}
}{% else
\setlength{\parindent}{0pt}
\setlength{\parskip}{6pt plus 2pt minus 1pt}
}
\setlength{\emergencystretch}{3em}  % prevent overfull lines
\providecommand{\tightlist}{%
  \setlength{\itemsep}{0pt}\setlength{\parskip}{0pt}}
\setcounter{secnumdepth}{0}
% Redefines (sub)paragraphs to behave more like sections
\ifx\paragraph\undefined\else
\let\oldparagraph\paragraph
\renewcommand{\paragraph}[1]{\oldparagraph{#1}\mbox{}}
\fi
\ifx\subparagraph\undefined\else
\let\oldsubparagraph\subparagraph
\renewcommand{\subparagraph}[1]{\oldsubparagraph{#1}\mbox{}}
\fi

%%% Use protect on footnotes to avoid problems with footnotes in titles
\let\rmarkdownfootnote\footnote%
\def\footnote{\protect\rmarkdownfootnote}

%%% Change title format to be more compact
\usepackage{titling}

% Create subtitle command for use in maketitle
\newcommand{\subtitle}[1]{
  \posttitle{
    \begin{center}\large#1\end{center}
    }
}

\setlength{\droptitle}{-2em}
  \title{}
  \pretitle{\vspace{\droptitle}}
  \posttitle{}
  \author{}
  \preauthor{}\postauthor{}
  \date{}
  \predate{}\postdate{}


\begin{document}

\section{{[}DRAFT{]} Analysis of Software and Data Carpentry's Pre- and
Post-Workshop
Surveys}\label{draft-analysis-of-software-and-data-carpentrys-pre--and-post-workshop-surveys}

\subsubsection{Authors: The Carpentries Assessment
Team}\label{authors-the-carpentries-assessment-team}

\subsubsection{Published: July 2018}\label{published-july-2018}

\subsubsection{Overview}\label{overview}

Since November 17, 2015, Software and Data Carpentry have collected
information on learner demographics, perception of tools and confidence
in working with data. As we continue in our goal to streamline processes
as The Carpentries, the Assessment Team completed an analysis of the
pre- and post-workshop surveys for both Software and Data Carpentry. The
goal of this analysis is to understand the impact our workshops are
having on learners, and how we can improve our surveys and assessment
infrastructure. This report covers the workshops from November 17, 2015
to May 21, 2018 for Software Carpentry, and from August 07, 2017 to May
11, 2018 for Data Carpentry.

As an overview, 1259 and 852 learners have responded to Data Carpentry's
pre- and post-workshop surveys respectively, while 14154 and 6458 have
responded to Software Carpentry's.

This report includes the following:

\begin{itemize}
\tightlist
\item
  Motivation for Attending Carpentries Workshops
\item
  Workshop Type and Perception of Workshop Environment/Experience
\item
  Effect of Workshops on Learner's Self-Reported Perspectives, Skills,
  and Confidence
\item
  Ability to Perform Computing Tasks
\item
  Demographics
\item
  Summary
\end{itemize}

\subsubsection{Motivation for Attending Carpentries
Workshops}\label{motivation-for-attending-carpentries-workshops}

Learners attend Carpentries workshops for many reasons. Data Carpentry's
workshops are domain-specific and focus on the fundamental data skills
needed to conduct research. Data Carpentry's Ecology and Social Sciences
curricula begin with a lesson on data organization and includes data
cleaning with OpenRefine. From there, learners spend time learning a
base programming language, either Python or R, to manipulate and
visualize data.

Data Carpentry's Genomics curriculum also includes programming, however
the focus of this curriculum is on best practices for organization of
bioinformatics projects and data, use of command line utilities and
tools to analyze sequence quality and perform variant calling, and
connecting to and using cloud computing.

\paragraph{Why are learners participating in our
workshops?}\label{why-are-learners-participating-in-our-workshops}

\begin{longtable}[]{@{}lrr@{}}
\toprule
Why learners attend Data Carpentry workshops? (n = 1236) & n &
\%\tabularnewline
\midrule
\endhead
To learn skills that I can apply to my current work & 1058 &
85.6\tabularnewline
To learn skills that I can apply to my work in the future & 960 &
77.7\tabularnewline
To learn skills that will help me get a job & 445 & 36.0\tabularnewline
As a requirement for my program/current position & 103 &
8.3\tabularnewline
\bottomrule
\end{longtable}

85.6\% of Data Carpentry learners attend workshops to learn skills they
can apply to their work in the future.

Software Carpentry workshops teach automation with the Unix shell, a
tool that allows users to run commands interactively or by scripting. In
Software Carpentry workshops, version control of source code with Git
and GitHub are also taught for learners to learn facilitating
contribution and collaboration on online repositories. Programming in R
or Python is also taught. Software Carpentry's curriculum teaches basic
lab skills for scientific computing.

\begin{longtable}[]{@{}lrr@{}}
\toprule
Why learners attend Software Carpentry Workshops? (n = 613) & n &
\%\tabularnewline
\midrule
\endhead
To cover new/additional topics & 457 & 74.6\tabularnewline
To refresh/review skills & 339 & 55.3\tabularnewline
To network & 78 & 12.7\tabularnewline
To become a Software Carpentry helper/instructor & 39 &
6.4\tabularnewline
To help host/run a workshop & 37 & 6.0\tabularnewline
\bottomrule
\end{longtable}

\begin{longtable}[]{@{}lrr@{}}
\toprule
Software Carpentry 1st Time Learners & n & \%\tabularnewline
\midrule
\endhead
Yes & 11495 & 81.2\tabularnewline
Didn't answer & 1963 & 13.9\tabularnewline
No & 696 & 4.9\tabularnewline
\bottomrule
\end{longtable}

Compared to Data Carpentry's learners, Software Carpentry's tend to have
more experience with the tools covered in the workshops, and learners
come to learn new and/or additional topics (74.6\%). It's also
interesting to note that 81.2\% of Software Carpentry respondents are
first-time attendees.

\paragraph{What is the current level of satisfaction of the data
management practices of our learners before attending our
workshops?}\label{what-is-the-current-level-of-satisfaction-of-the-data-management-practices-of-our-learners-before-attending-our-workshops}

\begin{longtable}[]{@{}lrr@{}}
\toprule
\begin{minipage}[b]{0.80\columnwidth}\raggedright\strut
Data Carpentry Learners satisfaction with current data management
practices\strut
\end{minipage} & \begin{minipage}[b]{0.05\columnwidth}\raggedleft\strut
n\strut
\end{minipage} & \begin{minipage}[b]{0.06\columnwidth}\raggedleft\strut
\%\strut
\end{minipage}\tabularnewline
\midrule
\endhead
\begin{minipage}[t]{0.80\columnwidth}\raggedright\strut
Unsatisfied\strut
\end{minipage} & \begin{minipage}[t]{0.05\columnwidth}\raggedleft\strut
408\strut
\end{minipage} & \begin{minipage}[t]{0.06\columnwidth}\raggedleft\strut
32.4\strut
\end{minipage}\tabularnewline
\begin{minipage}[t]{0.80\columnwidth}\raggedright\strut
Neutral\strut
\end{minipage} & \begin{minipage}[t]{0.05\columnwidth}\raggedleft\strut
405\strut
\end{minipage} & \begin{minipage}[t]{0.06\columnwidth}\raggedleft\strut
32.2\strut
\end{minipage}\tabularnewline
\begin{minipage}[t]{0.80\columnwidth}\raggedright\strut
Satisfied\strut
\end{minipage} & \begin{minipage}[t]{0.05\columnwidth}\raggedleft\strut
173\strut
\end{minipage} & \begin{minipage}[t]{0.06\columnwidth}\raggedleft\strut
13.7\strut
\end{minipage}\tabularnewline
\begin{minipage}[t]{0.80\columnwidth}\raggedright\strut
Very unsatisfied\strut
\end{minipage} & \begin{minipage}[t]{0.05\columnwidth}\raggedleft\strut
100\strut
\end{minipage} & \begin{minipage}[t]{0.06\columnwidth}\raggedleft\strut
7.9\strut
\end{minipage}\tabularnewline
\begin{minipage}[t]{0.80\columnwidth}\raggedright\strut
Not sure\strut
\end{minipage} & \begin{minipage}[t]{0.05\columnwidth}\raggedleft\strut
75\strut
\end{minipage} & \begin{minipage}[t]{0.06\columnwidth}\raggedleft\strut
6.0\strut
\end{minipage}\tabularnewline
\begin{minipage}[t]{0.80\columnwidth}\raggedright\strut
Not applicable\strut
\end{minipage} & \begin{minipage}[t]{0.05\columnwidth}\raggedleft\strut
60\strut
\end{minipage} & \begin{minipage}[t]{0.06\columnwidth}\raggedleft\strut
4.8\strut
\end{minipage}\tabularnewline
\begin{minipage}[t]{0.80\columnwidth}\raggedright\strut
Very satisfied\strut
\end{minipage} & \begin{minipage}[t]{0.05\columnwidth}\raggedleft\strut
22\strut
\end{minipage} & \begin{minipage}[t]{0.06\columnwidth}\raggedleft\strut
1.7\strut
\end{minipage}\tabularnewline
\begin{minipage}[t]{0.80\columnwidth}\raggedright\strut
Didn't answer\strut
\end{minipage} & \begin{minipage}[t]{0.05\columnwidth}\raggedleft\strut
16\strut
\end{minipage} & \begin{minipage}[t]{0.06\columnwidth}\raggedleft\strut
1.3\strut
\end{minipage}\tabularnewline
\bottomrule
\end{longtable}

\includegraphics[width=720]{figures/dc-satisfaction-level-plot-1}

The majority (72.5\%) of Data Carpentry's respondents are either
unsatisfied or feel neutral about being satisfied with their current
data management practices. By data management practices, we include
behaviors such as keeping your raw data raw, reusing code, and using
databases, queries, and scripts to manage large datasets.

\paragraph{How often do our workshop participants program before
attending our
workshops?}\label{how-often-do-our-workshop-participants-program-before-attending-our-workshops}

\begin{longtable}[]{@{}lrr@{}}
\toprule
Data Carpentry Learners current programming usage & n &
\%\tabularnewline
\midrule
\endhead
Never & 314 & 24.9\tabularnewline
Didn't answer & 276 & 21.9\tabularnewline
Several times per year & 168 & 13.3\tabularnewline
Less than once per year & 151 & 12.0\tabularnewline
Weekly & 131 & 10.4\tabularnewline
Daily & 118 & 9.4\tabularnewline
Monthly & 101 & 8.0\tabularnewline
\bottomrule
\end{longtable}

In terms of current programming usage, 36.9\% of learners either never
use programming, or use programming less than once per year, but no more
than several times per year. Only 9.4\% program on a daily basis. This
is no surprise, as Data Carpentry workshops tend to attract novices.

\begin{longtable}[]{@{}lrr@{}}
\toprule
How respondents find out about Data Carpentry workshops (n =1211) & n &
\%\tabularnewline
\midrule
\endhead
Received an email about the workshop & 802 & 66.2\tabularnewline
My friend/colleague told me about it & 324 & 26.8\tabularnewline
My advisor/supervisor told me about it & 244 & 20.1\tabularnewline
Read about it in a newsletter or university web site & 85 &
7.0\tabularnewline
Other web site & 20 & 1.7\tabularnewline
Twitter or other social media & 17 & 1.4\tabularnewline
\bottomrule
\end{longtable}

\begin{longtable}[]{@{}lrr@{}}
\toprule
\begin{minipage}[b]{0.78\columnwidth}\raggedright\strut
How respondents find out about Software Carpentry workshops (n =
10075)\strut
\end{minipage} & \begin{minipage}[b]{0.06\columnwidth}\raggedleft\strut
n\strut
\end{minipage} & \begin{minipage}[b]{0.06\columnwidth}\raggedleft\strut
\%\strut
\end{minipage}\tabularnewline
\midrule
\endhead
\begin{minipage}[t]{0.78\columnwidth}\raggedright\strut
Institution mailing list or flyer\strut
\end{minipage} & \begin{minipage}[t]{0.06\columnwidth}\raggedleft\strut
5143\strut
\end{minipage} & \begin{minipage}[t]{0.06\columnwidth}\raggedleft\strut
51.0\strut
\end{minipage}\tabularnewline
\begin{minipage}[t]{0.78\columnwidth}\raggedright\strut
Friend/colleague\strut
\end{minipage} & \begin{minipage}[t]{0.06\columnwidth}\raggedleft\strut
4810\strut
\end{minipage} & \begin{minipage}[t]{0.06\columnwidth}\raggedleft\strut
47.7\strut
\end{minipage}\tabularnewline
\begin{minipage}[t]{0.78\columnwidth}\raggedright\strut
Conference/meeting/seminar\strut
\end{minipage} & \begin{minipage}[t]{0.06\columnwidth}\raggedleft\strut
637\strut
\end{minipage} & \begin{minipage}[t]{0.06\columnwidth}\raggedleft\strut
6.3\strut
\end{minipage}\tabularnewline
\begin{minipage}[t]{0.78\columnwidth}\raggedright\strut
Our website\strut
\end{minipage} & \begin{minipage}[t]{0.06\columnwidth}\raggedleft\strut
356\strut
\end{minipage} & \begin{minipage}[t]{0.06\columnwidth}\raggedleft\strut
3.5\strut
\end{minipage}\tabularnewline
\begin{minipage}[t]{0.78\columnwidth}\raggedright\strut
Funding organization or program officer\strut
\end{minipage} & \begin{minipage}[t]{0.06\columnwidth}\raggedleft\strut
355\strut
\end{minipage} & \begin{minipage}[t]{0.06\columnwidth}\raggedleft\strut
3.5\strut
\end{minipage}\tabularnewline
\begin{minipage}[t]{0.78\columnwidth}\raggedright\strut
Social Media (Twitter, Facebook, etc.)\strut
\end{minipage} & \begin{minipage}[t]{0.06\columnwidth}\raggedleft\strut
293\strut
\end{minipage} & \begin{minipage}[t]{0.06\columnwidth}\raggedleft\strut
2.9\strut
\end{minipage}\tabularnewline
\begin{minipage}[t]{0.78\columnwidth}\raggedright\strut
Journal or publication\strut
\end{minipage} & \begin{minipage}[t]{0.06\columnwidth}\raggedleft\strut
37\strut
\end{minipage} & \begin{minipage}[t]{0.06\columnwidth}\raggedleft\strut
0.4\strut
\end{minipage}\tabularnewline
\bottomrule
\end{longtable}

Data Carpentry and Software Carpentry workshop participants often find
out about our workshops through institution mailing lists (66.2\% and
51\% respectively). However, ``word-of-mouth'' also plays a significant
role in populating the workshops.

In summary, both Data and Software Carpentry workshop respondents attend
workshops to learn about or improve upon their current data management
and analysis skills.

\subsubsection{Workshop Type and Perception of Workshop
Environment/Experience}\label{workshop-type-and-perception-of-workshop-environmentexperience}

\includegraphics[width=720]{figures/dc-workshop-type-plot-1}

\begin{longtable}[]{@{}lrr@{}}
\toprule
Data Carpentry: Language Covered in Workshops & n & \%\tabularnewline
\midrule
\endhead
R & 614 & 72.1\tabularnewline
Python & 118 & 13.8\tabularnewline
Didn't answer & 58 & 6.8\tabularnewline
Neither & 57 & 6.7\tabularnewline
I don't know./I don't remember. & 5 & 0.6\tabularnewline
\bottomrule
\end{longtable}

As previously mentioned, Data Carpentry workshops are domain specific,
and curricula include Ecology, Genomics, Geospatial, Social Sciences,
and Reproducible Research. 72.1\% of respondents learned R in their
workshop, while 13.8\% learned Python.

\subsection{Workshop Environment}\label{workshop-environment}

The Carpentries is committed to making participation in our workshops a
harassment-free experience for everyone, regardless of who you are,
where you come from, or your experience with the tools we teach. We
establish norms for interaction by having, discussing, and enforcing a
Code of Conduct such that our workshops provide open and inclusive
learning environments. 79\% of Data Carpentry respondents either agree
or strongly agree that they felt comfortable learning in their workshop
environment, and 87.1\% of Software Carpentry's respondents agreed or
strongly agreed the workshop atmosphere was welcoming.

\includegraphics[width=720]{figures/dc-post-workshop-comfortable-environment-1}

\includegraphics[width=720]{figures/swc-post-workshop-environment-1}

Data Carpentry respondents were asked to rate their level of agreement
with several statements regarding their instructor's knowledge,
instructional method, and enthusiasm. Their responses are in the figure
below, and axis labels correspond to the statements as follows:

\begin{itemize}
\tightlist
\item
  \textbf{Instructors Knowledge}: The instructors were knowledgeable
  about the material being taught.
\item
  \textbf{Instructors Interacting}: I felt comfortable interacting with
  the instructors.
\item
  \textbf{Instructors Enthusiastic}: The instructors were enthusiastic
  about the workshop.
\item
  \textbf{Instructors Clear Answers}: I was able to get clear answers to
  my questions from the instructors.
\end{itemize}

\includegraphics[width=720]{figures/dc-perception-instructors-heatmap-1}

The largest impact we see is that 96.2\% of respondents said they felt
comfortable interacting with the instructors. We know that our
instructors \emph{are} the reason why our workshops are so well
received. It's also great to see that 94.8\% and 96.7\% of respondents
felt our instructors were knowledgeable about the material being taught,
and were enthusiastic about the workshop, respectively. We would like to
explore what training would help our instructors so that the percentage
of respondents who felt they were able to get clear answers to their
questions from the instructors would increase.

Software Carpentry respondents were asked to rate how they felt
instructors and helpers worked as a team based on the following
criteria:

\begin{itemize}
\tightlist
\item
  \textbf{Considerate}: Instructors/Helpers were considerate.
\item
  \textbf{Enthusiastic}: Instructors/Helpers were enthusiastic.
\item
  \textbf{Clear Answers}: Instructors/Helpers gave clear answers to your
  questions.
\item
  \textbf{Communicators}: Instructors/Helpers were good communicators.
\end{itemize}

The two Likert plots below provide an analysis of respondent's answers
for both instructors and helpers.

\includegraphics[width=720]{figures/swc-perception-instructors-1}

\includegraphics[width=720]{figures/swc-perception-helpers-1}

From the figures above, we see that Software Carpentry instructors and
helpers are considerate, enthusiastic, give clear answers to questions,
and are good communicators. As a whole, our instructors work as a team
and are successful in creating a warm and welcoming workshop
environment.

One of the goals for Data Carpentry's lessons is that learners are able
to immediately apply what they learned at the workshop. The figure below
shows that 65.2\% either agree or strongly agree that they're able to
immediately apply what they learned.

\includegraphics[width=720]{figures/dc-skill-applicability-plot-1}

As the majority of Software Carpentry learners attend workshops to learn
new skills it is great to see that 47.2\% of learners either learned
mostly or all new information during the workshop.

\includegraphics[width=720]{figures/swc-new-information-plot-1}

\subsubsection{Workshop Experience}\label{workshop-experience}

\begin{longtable}[]{@{}lrr@{}}
\toprule
Data Carpentry Respondents Having Accessibility Issues & n &
\%\tabularnewline
\midrule
\endhead
No & 654 & 76.8\tabularnewline
Didn't answer & 119 & 14.0\tabularnewline
Yes & 79 & 9.3\tabularnewline
\bottomrule
\end{longtable}

We want to be proactive in ensuring learners have access to whatever
they need to participate in a workshop. Both Data and Software Carpentry
learners are asked to inform workshop organizers if there is anything
they need that would make their workshop experience better. Data
Carpentry's respondents were asked if they had accessibility issues, and
9.3\% reported they did. After reading the open-ended responses, we see
that the issues were related to not being able to hear and/or see in the
back of the room.

We use the \href{https://en.wikipedia.org/wiki/Net_Promoter}{Net
Promoter Score} to measure learners' likelihood of recommending
workshops to a friend or colleague. The scoring for this question is on
a 0 to 100 scale. Respondents scoring from 0 to 64 are labeled
\emph{Detractors}, and are believed to be less likely to recommend a
workshop. Those who respond with a score of 85 to 100 are called
\emph{Promoters}, and are considered likely to recommend a workshop.
Respondents between 65 and 84 are labeled \emph{Passives}, and their
behavior falls in the middle of Promoters and Detractors.

\begin{longtable}[]{@{}lrr@{}}
\toprule
Data Carpentry Net Promoter Score & n & \%\tabularnewline
\midrule
\endhead
Promoter & 565 & 66.3\tabularnewline
Passive & 131 & 15.4\tabularnewline
Didn't answer & 124 & 14.6\tabularnewline
Detractor & 32 & 3.8\tabularnewline
\bottomrule
\end{longtable}

77.6\% of Data Carpentry respondants who answered this question are
promoters (i.e.~would recommend a workshop).

For Software Carpentry respondents who answered this questions, 56.9\%
are promoters.

In summary, Data and Software Carpentry workshops provide a warm and
welcoming environment whether learners are brand new to programming or
have some experience. Attendees are recommending workshops to their
friends and colleagues, and we know that our instructors and helpers are
the major reason why.

\subsubsection{Effect of Workshops on Learners Self-Reported
Perspectives: Skills \&
Confidence}\label{effect-of-workshops-on-learners-self-reported-perspectives-skills-confidence}

Learners were asked to rate their level of agreement with the following
statements related to Data Carpentry's workshop goals and learning
objectives. The figure below provides a visual representation of their
responses, comparing them before the workshop and after the workshop.
Axis labels and the corresponding question are as follows:

\begin{itemize}
\tightlist
\item
  \textbf{Write Program}: I can write a small program/script/macro to
  solve a problem in my own work.
\item
  \textbf{Search Online}: I know how to search for answers to my
  technical questions online.
\item
  \textbf{Raw Data}: Having access to the original, raw data is
  important to be able to repeat an analysis.
\item
  \textbf{Programming Efficient}: Using a programming language (like R
  or Python) can make me more efficient at working with data.
\item
  \textbf{Programming Confident}: I am confident in my ability to make
  use of programming software to work with data.
\item
  \textbf{Overcome Problem}: While working on a programming project, if
  I get stuck, I can find ways of overcoming the problem.
\item
  \textbf{Analyses Easier}: Using a programming language (like R or
  Python) can make my analyses easier to reproduce.
\end{itemize}

\includegraphics[width=720]{figures/dc-paired-data-mode-1}

\includegraphics[width=720]{figures/dc-paired-data-mean-1}

As the scoring for the above factors is ordinal from strongly disagree
(1) to strongly agree (5), we show the mode (most frequent responses)
for respondents' before the workshop, and after the workshop. If there
is only one point on the plot it is because the mode for pre- and
post-workshop responses was the same. The comparison above is paired,
meaning, we are comparing those who provided us with a unique identifier
and who completed both the pre- and post-workshop survey. This figure
includes XX respondses.

In the figures below we show another representation of the pre- and
post-comparison of respondents skills and perspectives. The figures
below include the data for all learners, not only those who provided a
unique identifier \emph{and} took both the pre- and post-workshop
surveys. What we see is a shift in the distribution of for each factor,
meaning, respondents' self-reported confidence and ability shifted in a
positive directions.

\includegraphics[width=720]{figures/dc-paired-distribution-comparison-1}

Another representation of the positive shift in distribution is provided
below.

\includegraphics[width=720]{figures/dc-paired-distribution-shift-1}

The neutral centered graphs below provide an even clearer picture of the
shift in respondents self-reported confidence and skills.

It is interesting to see the shift in neutrality between the
pre-workshop scores and post-workshops scores, especially for
\emph{Programming Efficient}. There was a higher percentage of learners
beginning the workshop who felt programming with R or Python can make
them more efficient at working with data. Contrarily, confidence in
using programming to work with data increased from XX\% to xx\%.

\includegraphics[width=720]{figures/dc-tools-perception-1}

Software Carpentry Respondents were asked to tell us about their
experience with these topics before the workshop:

\begin{itemize}
\tightlist
\item
  R
\item
  Unix Shell
\item
  SQL
\item
  Python
\item
  Version Control with Git
\end{itemize}

\includegraphics[width=720]{figures/swc-pre-tools-1}

From the figure we see that XX\% and less had extensive knowledge of the
topics covered in their workshop.

\includegraphics[width=720]{figures/swc-knowledge-change-1}

The following is a comparison of Software Carpentry Respondents'
knowledge about the tools before compared to after the workshop. We see
clearly that after the workshop, respondents' knowledge of Git, Python,
R, and the Unix Shell increased a great deal.

\subsubsection{Respondent Ability to Perform Computing
Tasks}\label{respondent-ability-to-perform-computing-tasks}

Motivation is important, but being confident in your ability to complete
specific computing tasks is an equally important goal of Software
Carpentry. The grid below shows respondents' self-reported ability to
complete tasks including:

\begin{itemize}
\tightlist
\item
  Using pipes to connect shell commands
\item
  Writing a `for loop' to automate tasks\\
\item
  Initializing a repository with git
\item
  Writing a function
\item
  Importing a library or package in R or Python
\item
  Writing a unit test in Python or R
\item
  Writing an SQL query
\end{itemize}

It also provides their self-reported level of confidence in being able
to complete the tasks above after completing the workshop.

These figures tell us that, before the workshop, between XX\% and XX\%
of the respondents did not feel they could initialize a repository in
Git, write a `for loop' to automate tasks, use pipes to connect shell
commands, write a SQL query, and/or write a unit test in R or Python.
XX\% of learners felt their confidence increased greatly with respect to
importing a library or package in R or Python. We consider this
significant as it is one of the fundamental skills that allows learners
to be successful in the other areas mentioned above.

In summary, respondents experienced increased confidence in their
ability to perform specific computing tasks and solve problems, or at
least search for answers to problems, as a result of participating in
Software and Data Carpentry workshops.

\includegraphics[width=720]{figures/swc-ability-confidence-1}

\subsubsection{Demographics}\label{demographics}

The Carpentries is a global community that has recognized the importance
of bringing people to data through high-impact trainings. Thought the
majority of Data Carpentry respondents attended a workshop in the United
States of America (45.3\%), we see an increase in workshops in places
like Ethiopia (3.7\%), Switzerland (0.6\%), and India (0.1\%).

\includegraphics[width=720]{figures/dc-country-workshop-plot-1}

\begin{longtable}[]{@{}lrr@{}}
\toprule
Software Carpentry Workshops in US & n & \%\tabularnewline
\midrule
\endhead
Yes & 6468 & 45.7\tabularnewline
No & 4640 & 32.8\tabularnewline
Didn't answer & 3046 & 21.5\tabularnewline
\bottomrule
\end{longtable}

In Software Carpentry's pre-workshop survey, respondents are asked
whether or not their workshop takes place in the United States. 45.7\%
of respondents attended a U.S. workshop.

\begin{longtable}[]{@{}lrr@{}}
\toprule
Data Carpentry's Respondents by Discipline & n & \%\tabularnewline
\midrule
\endhead
Life Sciences & 444 & 37.4\tabularnewline
Agricultural or Environmental Sciences & 307 & 25.9\tabularnewline
Bioinformatics/Genomics & 292 & 24.6\tabularnewline
Biomedical/Health Sciences & 288 & 24.3\tabularnewline
Social Sciences & 122 & 10.3\tabularnewline
Mathematics or Statistics & 101 & 8.5\tabularnewline
Earth Sciences & 96 & 8.1\tabularnewline
Engineering & 91 & 7.7\tabularnewline
Computer Science & 88 & 7.4\tabularnewline
Business/Economics & 57 & 4.8\tabularnewline
Physical Sciences & 53 & 4.5\tabularnewline
Humanities & 53 & 4.5\tabularnewline
Library Sciences & 28 & 2.4\tabularnewline
\bottomrule
\end{longtable}

As previously mentioned, Data Carpentry's curricula is domain specific
to Ecology, Genomics, Geospatial, and the Social Sciences. We see this
in the distribution of respondents by discipline. 37.4\% are in the Life
Science, while 25.9\%, 24.6\%, and 24.3\% are in Agricultural or
Environmental Sciences, Bioinformatics/Genomics, and Biomedical/Health
Sciences, respectively.

\begin{longtable}[]{@{}lrr@{}}
\toprule
\begin{minipage}[b]{0.81\columnwidth}\raggedright\strut
Software Carpentry's Respondents by Discipline\strut
\end{minipage} & \begin{minipage}[b]{0.05\columnwidth}\raggedleft\strut
n\strut
\end{minipage} & \begin{minipage}[b]{0.05\columnwidth}\raggedleft\strut
\%\strut
\end{minipage}\tabularnewline
\midrule
\endhead
\begin{minipage}[t]{0.81\columnwidth}\raggedright\strut
Life Science - Organismal/systems (ecology, botany, zoology,
microbiology, neuroscience)\strut
\end{minipage} & \begin{minipage}[t]{0.05\columnwidth}\raggedleft\strut
2694\strut
\end{minipage} & \begin{minipage}[t]{0.05\columnwidth}\raggedleft\strut
28.1\strut
\end{minipage}\tabularnewline
\begin{minipage}[t]{0.81\columnwidth}\raggedright\strut
Life Sciences (Genetics, genomics, bioinformatics )\strut
\end{minipage} & \begin{minipage}[t]{0.05\columnwidth}\raggedleft\strut
2680\strut
\end{minipage} & \begin{minipage}[t]{0.05\columnwidth}\raggedleft\strut
27.9\strut
\end{minipage}\tabularnewline
\begin{minipage}[t]{0.81\columnwidth}\raggedright\strut
Mathematics/statistics\strut
\end{minipage} & \begin{minipage}[t]{0.05\columnwidth}\raggedleft\strut
940\strut
\end{minipage} & \begin{minipage}[t]{0.05\columnwidth}\raggedleft\strut
9.8\strut
\end{minipage}\tabularnewline
\begin{minipage}[t]{0.81\columnwidth}\raggedright\strut
Physics\strut
\end{minipage} & \begin{minipage}[t]{0.05\columnwidth}\raggedleft\strut
801\strut
\end{minipage} & \begin{minipage}[t]{0.05\columnwidth}\raggedleft\strut
8.4\strut
\end{minipage}\tabularnewline
\begin{minipage}[t]{0.81\columnwidth}\raggedright\strut
Planetary sciences (geology, climatology, oceanography, etc.)\strut
\end{minipage} & \begin{minipage}[t]{0.05\columnwidth}\raggedleft\strut
786\strut
\end{minipage} & \begin{minipage}[t]{0.05\columnwidth}\raggedleft\strut
8.2\strut
\end{minipage}\tabularnewline
\begin{minipage}[t]{0.81\columnwidth}\raggedright\strut
Civil, mechanical, chemical, or nuclear engineering\strut
\end{minipage} & \begin{minipage}[t]{0.05\columnwidth}\raggedleft\strut
692\strut
\end{minipage} & \begin{minipage}[t]{0.05\columnwidth}\raggedleft\strut
7.2\strut
\end{minipage}\tabularnewline
\begin{minipage}[t]{0.81\columnwidth}\raggedright\strut
Medicine and/or Pharmacy\strut
\end{minipage} & \begin{minipage}[t]{0.05\columnwidth}\raggedleft\strut
684\strut
\end{minipage} & \begin{minipage}[t]{0.05\columnwidth}\raggedleft\strut
7.1\strut
\end{minipage}\tabularnewline
\begin{minipage}[t]{0.81\columnwidth}\raggedright\strut
Social sciences\strut
\end{minipage} & \begin{minipage}[t]{0.05\columnwidth}\raggedleft\strut
591\strut
\end{minipage} & \begin{minipage}[t]{0.05\columnwidth}\raggedleft\strut
6.2\strut
\end{minipage}\tabularnewline
\begin{minipage}[t]{0.81\columnwidth}\raggedright\strut
Chemistry\strut
\end{minipage} & \begin{minipage}[t]{0.05\columnwidth}\raggedleft\strut
574\strut
\end{minipage} & \begin{minipage}[t]{0.05\columnwidth}\raggedleft\strut
6.0\strut
\end{minipage}\tabularnewline
\begin{minipage}[t]{0.81\columnwidth}\raggedright\strut
Economics/business\strut
\end{minipage} & \begin{minipage}[t]{0.05\columnwidth}\raggedleft\strut
481\strut
\end{minipage} & \begin{minipage}[t]{0.05\columnwidth}\raggedleft\strut
5.0\strut
\end{minipage}\tabularnewline
\begin{minipage}[t]{0.81\columnwidth}\raggedright\strut
Psychology\strut
\end{minipage} & \begin{minipage}[t]{0.05\columnwidth}\raggedleft\strut
417\strut
\end{minipage} & \begin{minipage}[t]{0.05\columnwidth}\raggedleft\strut
4.3\strut
\end{minipage}\tabularnewline
\begin{minipage}[t]{0.81\columnwidth}\raggedright\strut
Library and information science\strut
\end{minipage} & \begin{minipage}[t]{0.05\columnwidth}\raggedleft\strut
373\strut
\end{minipage} & \begin{minipage}[t]{0.05\columnwidth}\raggedleft\strut
3.9\strut
\end{minipage}\tabularnewline
\begin{minipage}[t]{0.81\columnwidth}\raggedright\strut
High performance computing\strut
\end{minipage} & \begin{minipage}[t]{0.05\columnwidth}\raggedleft\strut
360\strut
\end{minipage} & \begin{minipage}[t]{0.05\columnwidth}\raggedleft\strut
3.8\strut
\end{minipage}\tabularnewline
\begin{minipage}[t]{0.81\columnwidth}\raggedright\strut
Humanities\strut
\end{minipage} & \begin{minipage}[t]{0.05\columnwidth}\raggedleft\strut
318\strut
\end{minipage} & \begin{minipage}[t]{0.05\columnwidth}\raggedleft\strut
3.3\strut
\end{minipage}\tabularnewline
\begin{minipage}[t]{0.81\columnwidth}\raggedright\strut
Education\strut
\end{minipage} & \begin{minipage}[t]{0.05\columnwidth}\raggedleft\strut
264\strut
\end{minipage} & \begin{minipage}[t]{0.05\columnwidth}\raggedleft\strut
2.8\strut
\end{minipage}\tabularnewline
\begin{minipage}[t]{0.81\columnwidth}\raggedright\strut
Space sciences\strut
\end{minipage} & \begin{minipage}[t]{0.05\columnwidth}\raggedleft\strut
161\strut
\end{minipage} & \begin{minipage}[t]{0.05\columnwidth}\raggedleft\strut
1.7\strut
\end{minipage}\tabularnewline
\bottomrule
\end{longtable}

Software Carpentry's respondent base also has a majority Life Sciences
base, however we also see representation from those working in
Psychology, High Performance Computing, and Chemistry.

\begin{longtable}[]{@{}lrr@{}}
\toprule
Data Carpentry's Respondents by Position & n & \%\tabularnewline
\midrule
\endhead
Graduate Student & 592 & 49.7\tabularnewline
Research Staff & 200 & 16.8\tabularnewline
Postdoctoral Researcher & 183 & 15.4\tabularnewline
Faculty & 101 & 8.5\tabularnewline
Government Employee & 80 & 6.7\tabularnewline
Industry Employee & 49 & 4.1\tabularnewline
Undergraduate Student & 48 & 4.0\tabularnewline
Management/Administrator & 20 & 1.7\tabularnewline
Retired/Not Employed & 18 & 1.5\tabularnewline
\bottomrule
\end{longtable}

As many of The Carpentries' workshops are hosted on college campuses and
other research-based communities, there is no surprise that the majority
of respondents are Graduate Students (49.7\% - DC, 35.4\% - SWC),
Research Staff (16.8\% - DC,9.6\% - SWC), and Postdoctoral Researchers
(1.5\% - DC, 12.3\% - SWC).

\includegraphics[width=720]{figures/dc-status-plot-1}

\includegraphics[width=720]{figures/swc-status-plot-1}

\begin{longtable}[]{@{}lrr@{}}
\toprule
Operating System Respondents Use in Data Carpentry Workshops & n &
\%\tabularnewline
\midrule
\endhead
Windows & 661 & 53.3\tabularnewline
Apple/Mac OS & 512 & 41.3\tabularnewline
UNIX/Linux & 50 & 4.0\tabularnewline
Not sure & 17 & 1.4\tabularnewline
\bottomrule
\end{longtable}

In our workshops, we recommend that learners use their own machines. It
is important for learners to leave the workshop with their own machine
set up to do real work. Our instructors teach on three major platforms:
UNIX/Linux, Mac OS X, and Windows. We see a very close representation of
Windows (53.3\%) and Apple/Mac OS (41.3\%) users in our Data Carpentry
workshops, and even a few UNIX/Linux users (4\%).

Gender and racial/ethnic identity information is collected for U.S.
participants, as we are keen to increase the number of diverse
instructors and learners we serve. Understanding our demographic makeup
helps us to strategize about what communities to reach and what programs
to offer.

Currently, both Data (XX\%) and Software (XX\%) Carpentry see strong
representation from Women in the United States. Where we hope to improve
is in reaching the non-White audience, as less than XX\% of our
respondents are from communities historically underrepresented in the
science, technology, engineering, and mathematics (STEM) fields.

\begin{longtable}[]{@{}lrr@{}}
\toprule
Data Carpentry's U.S. Respondents' Gender Identity & n &
\%\tabularnewline
\midrule
\endhead
Female & 322 & 56.6\tabularnewline
Male & 223 & 39.2\tabularnewline
Didn't answer & 14 & 2.5\tabularnewline
Prefer not to answer & 8 & 1.4\tabularnewline
Transgender female & 2 & 0.4\tabularnewline
Transgender male & 0 & 0.0\tabularnewline
Gender variant/non-conforming & 0 & 0.0\tabularnewline
\bottomrule
\end{longtable}

\begin{longtable}[]{@{}lrr@{}}
\toprule
Data Carpentry's U.S. Respondents Racial/Ethnic Identity & n &
\%\tabularnewline
\midrule
\endhead
White & 316 & 58.1\tabularnewline
Asian & 152 & 27.9\tabularnewline
Hispanic or Latino(a) & 57 & 10.5\tabularnewline
I prefer not to say. & 28 & 5.1\tabularnewline
Black or African American & 25 & 4.6\tabularnewline
American Indian or Alaska Native & 4 & 0.7\tabularnewline
Native Hawaiian or Other Pacific Islander & 3 & 0.6\tabularnewline
\bottomrule
\end{longtable}

\begin{longtable}[]{@{}lrr@{}}
\toprule
Software Carpentry's U.S. Respondents' Gender Identity & n &
\%\tabularnewline
\midrule
\endhead
Male & 3111 & 49.0\tabularnewline
Female & 3107 & 48.9\tabularnewline
Prefer not to say & 115 & 1.8\tabularnewline
Other & 18 & 0.3\tabularnewline
\bottomrule
\end{longtable}

\begin{longtable}[]{@{}lrr@{}}
\toprule
Software Carpentry's U.S. Respondents' Racial/Ethnic Identity & n &
\%\tabularnewline
\midrule
\endhead
White / Caucasian & 3447 & 55.1\tabularnewline
Asian / Pacific Islander & 1552 & 24.8\tabularnewline
Hispanic or Latino & 387 & 6.2\tabularnewline
Prefer not to say & 374 & 6.0\tabularnewline
Black or African American & 241 & 3.9\tabularnewline
Multiple ethnicity / Other (please specify) & 221 & 3.5\tabularnewline
American Indian or Alaskan Native & 29 & 0.5\tabularnewline
Native Hawaiian or Other Pacific Islander & 5 & 0.1\tabularnewline
\bottomrule
\end{longtable}

\subsubsection{Summary}\label{summary}


\end{document}
